\chapter{Introduction}
\label{chap-1-intro}
\begin{ChapAbstract}
In this chapter, we talk about overview of our study, from remote sensing introduction, related work to our contributions. At the end, this thesis's outlines is also provided.

\end{ChapAbstract}


\section{Overview} 

Remote sensing, and satellite images in particular, has become an indispensable part in our daily life. Many different techniques can be used to apply on these kind of data, in order to monitor or measure phenomena found in the lithosphere, biosphere, hydrophere and atmostphere. With the help of mechanical devices, which are known as remote sensors, the huge and diverse amount of data was collected by many satellite. NASA, who has many public satellite programs, have been proved invaluable to the research community. For example, Landsat program is one of them where its images of the Earth surface since 1984 with high resolution (from 10m to 30m per pixel) have been made publicly  free of charge. Recently ESA (European Space Agency) has endeavored a similar but more ambitious program called Corpernicus in which a series of Sentinel satellites have been launched since 2014.

Remote sensing imagery has many applications in mapping land-use and cover, sols mapping, forestry, city mapping and monitoring, archaeological investigations, geomorphological surveying and so on. For example, foresters use aerial photographs for preparing forest cover maps, locating possible access roads, and measuring quantities of trees harvested. Specialized photography using color infrared film has also been used to detect disease and insect damage in forest trees. 

With more modern device nowadays, it is able to collect image data with better quality, higher resolution, and on multiple bands covering lands, waters, seas, and atmosphere. Our study is projected into those advances and to be expected to gain a better understanding the complex of water cycle under human impact, with help of Machine Learning.


\section{Introduction to Remote Sensing} % 2 pages

As we known, remote sensing imagery is the data that acquired from satellites on the sky. The uses of those data might be making surface map (for examples: Google Maps, Apple Map); creating maps of land surface temperature, reflectance and elevation; monitoring hydrology and land changes, etc. 

The Two of most popular between many kinds of satellites are optical-type and radar-type.

\subsection{Optical satellites}

Satellites with optical sensors provide images of the Earth over relatively large areas and are useful in the production of hydrology and vegetation maps. The sensors function in the optical part of wavelength spectrum, including visible, near infrared and short-wave infrared wavelengths. Satellite sensors commonly used for detailed mapping include Landsat, Sentinel-2, etc., with moderate resolution (resolution is approximately from 10 to 30 meters).

\begin{figure}
\centering
\includegraphics[width=0.7\linewidth]{figures/wavelengthL8.png}
\caption{Landsat 8 data specification. Source: USGS}
\end{figure}

Wavelengths of optical satellites are useful for distinguishing between forest types and other vegetation classes. Optical satellite data can be combined with laser data because the color information in optical satellite data can distinguish different vegetation types while laser data provides additional information about terrain or vegetation characteristics. 

\begin{figure}[h!]
\centering
\subfloat[]{
	\includegraphics[width=0.24\linewidth]{figures/trueColor.jpg}
} 
\subfloat[]{\includegraphics[width=0.24\linewidth]{figures/landwater.jpg}}
\subfloat[]{\includegraphics[width=0.24\linewidth]{figures/ndwi.jpg}} 
\subfloat[]{\includegraphics[width=0.24\linewidth]{figures/agriculture.jpg}}
\centering
\caption{Samples of bands combinations from Landsat 8 data files:
\textbf{(a)} True Color (B4, B3, B2), \textbf{(b)} Land/Water (B5, B6, B4), \textbf{(c)} Normalized difference water index(B3 - B5)(B3 + B5), \textbf{(d)} Agriculture(B6, B5, B2).}
\end{figure}

\subparagraph{Addition: The Landsat 8 Pre-Collection Quality Assessment (QA) band}

QA band is a part of Landsat 8 data files. Each pixel in the QA band contains integer that represent bit-packed combination of surface, atmosphere and sensor conditions that can affect overall usefulness of a given pixel. Depending on its pixel value (mostly depending on QA Bits), we can detect that pixel is a snow/ice, cloud or water (because these are mainly formed by water) or detect the cloud direction due to cirrus confidence. For more information about QA Band, see at: \href{https://landsat.usgs.gov/qualityband}{https://landsat.usgs.gov/qualityband}.

\begin{figure}
	\centering
	\includegraphics[width=0.32\textwidth]{figures/qaL8.jpg}
	\caption{Landsat 8 QA Band, Tri An Reservoir. Date taken: May 30, 2017}
\end{figure}

\subsection{Radar satellites}

Radar satellites can solve problem of satellites image on cloud days, and this is the biggest advantage of radar data over optical data. They are not affected by cloud because of its instrument specifications. For example, Sentinel-1 satellites use the C-Band Synthetic Aperture Radar (SAR). This colored Sentinel-1 SAR image is produced by showing the two polarisations (VV and VH i.e. vertical polarisation send for the radar signal and vertical or horizontal receive). 

\begin{figure}[h!]
	\centering
	\includegraphics[width=0.8\textwidth]{figures/sarImgVVS1.png}
	\caption{Sentinel-1 (band VV) image, Tri An Reservoir. Date taken: June 11, 2017}
\end{figure}	

With the initial goal of monitoring reservoirs, SAR images seem to be the swiss knife to our problem until it is put in the context of machine learning the hydrological cycle of reservoirs for prediction. A long times-series of satellite images is the prerequisite to detect such patterns and therefore when it comes to SAR images the data source is very scarce. The Sentinel-1 satellites coverage is just from 2014. Other SAR image sources are costly or nontrivial to access. This makes a strong motivation for us to achieve a robust  method that remove clouds from multi-spectral satellite images and make them useful for monitoring purpose. From this point of view, a long term data source such as of Landsat-5, 7, and 8 is very useful to understand the water cycle of reservoir which is often complexed by hydrological, economical, and political factors. Furthermore, multi-spectral images are useful in monitoring vegetation, which is closely related to the amount of water being used from reservoirs for irrigation. Ideally multi-spectral data can be combined with SAR data to provide the best temporal coverage and help to distinguish vegetation types while laser data provides additional information about water content in soil. 


\section{Related Work} % 2 pages

- Water body segmentation problems. +3 paper.
- Water Resources Monitoring problems and application. +3 paper
- Water body recover. +3 paper.
- Prediction problems

\section{Contributions} % 1 pages

- Combination: Detect anomaly point in order to detect point of time, that water body act anomaly season
- Short introduction of 3 part...

\section{Outlines} % 1 pages
- .....
