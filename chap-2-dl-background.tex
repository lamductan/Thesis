\chapter{Deep Learning Background}
\label{chap-2-dl-background}
\begin{ChapAbstract}
In this chapter, we present some background on machine learning and neural
networks in general, that will make this thesis relatively self-contained.
For a more thorough and slower-paced introduction we recommend the Deep
learnig book from Goodfellow et al \cite{dlbook}.
\end{ChapAbstract}

\section{Overview}
Machine learning is the domain of creating machines which are able to solving some specific tasks through learning from data. This is efficient in a variety of problems, consisting of time-series prediction, whose solutions are too difficult for a traditional software to deal with. However, what is exact learnt by machine in machine learning strategy? A commonly-cited definition is "A computer program is said to learn from experience \textit{E} with respect to some class of tasks \textit{T} and performance measure \textit{P}, if its performance at tasks in \textit{T}, as measured by \textit{P}, improves with experience \textit{E}" \cite{Mitchell:1997:ML}.

Deep learning uses deep neural networks to solve machine learning tasks. Neural networks include a set of neurons and connections. These neurons are managed as layers, and there are no connection between neurons in the same layer. A \textit{neuron} receives many inputs from predecessor-layer neurons and produces one output. Two special neuron's types are: inputs neurons, which are neurons in the first layer and outputs neurons, which are neurons in the last layer. A deep neural network usually has eight layers between input and output layer. Output of neuron $i$ can be transfered to input of neuron $j$, if there is a \textit{connection} between them. Each connection contains a scalar value $w$ called weight, which controls the strength of input signal by a multiplication with output of neuron $i$ before transfered to neuron $j$. Learning a neural network is finding the most appropriate weights through a process called \textit{training}.


\section{Supervised Learning}
Supervised learning is the class of learning problems where the desired output of the model on some training set is know and supplied by a supervisor. One example of this is the house's price prediction problem, where the learned model is function $\displaystyle f(\vx)$ that maps information of a house (area, position, architecture, prices at previous time, \dots) $\displaystyle \vx$ to a price in future $\displaystyle \vy$. In classification problem, the training set consists of a design matrix $\displaystyle \tX$ and a vector of scalars $\displaystyle \ry$ indicating the category of $\displaystyle \tX$. 

A typical supervised learning task consists of a training set $\displaystyle S$ of inputs-target pairs $\displaystyle (x, y)$, where $\displaystyle x$ is drawn from an input space $\displaystyle \tX$ and $\displaystyle y$ is drawn from an output space $\displaystyle \tY$, and a disjoint test set $\displaystyle T \sim D = (\tX x \tY)$. Let consider a function $\displaystyle f : X \rightarrow Y$ The goal of supervised learning is to find $\displaystyle f$ which minimizes the error between $\displaystyle f(x)$ and $\displaystyle y$, for all $\displaystyle (x,y) \in D$, or more formally, to find $\displaystyle f$ that:
\[ f^* = \argmin_{f \in F} E_{(x,y) \sim D}[L(f(x),y)] \label{equation_all_sim_in_D} \],
where $\displaystyle L$ is a scalar-value function representing the error of $\displaystyle f(x)$ and $\displaystyle y$. Although finding $\displaystyle f$ satisfied \eqref{equation_all_sim_in_D} is impossible (because we don't know explicitly what is $\displaystyle D$), we can switch to a simpler task by finding $\displaystyle f$, by using the i.i.d. assumption on $\displaystyle T$, such that:
\[ f^* = \argmin_{f \in F} E_{(x,y) \in T}[L(f(x), y)]\]
The i.i.d. assumption also works on the training set $\displaystyle S$, so we expect that:
\[ E_{(x,y) \in S}[L(f(x), y)] \sim E_{(x,y) \in T}[L(f(x), y)] \sim E_{(x,y) \sim D}[L(f(x),y) \]
Therefore, function $\displaystyle f$ can be found by searching a function $\displaystyle f$ over a class of function $\displaystyle F$ whose error $\displaystyle L$ on training set $\displaystyle S$ is as small as possible:
\[ f^* = \argmin_{f \in F} E_{(x,y) \in S}[L(f(x), y)] \]
The most important and painful step in solving a supervised learning problem is to find the most appropriate class of function $\displaystyle F$, because when we already find out the structure of $\displaystyle F$, we merely focus on searching the tuple of parameters $\displaystyle \theta$ of $\displaystyle f \in F$ such that:
\[ {\theta}^{*} = \argmin_{\theta}E_{(x,y) \in S}[L(f_{\theta}(x), y)] \label{equation_loss_in_train_set_of_f_theta} \]
and totally certain that the error on test set $\displaystyle T$ is adequately small enough. There is no explicit algorithm or rule to find such $\displaystyle F$, but following some previous research, %cite as Sutskever%
the size of $\displaystyle f$, or $|\theta|$, is ratio with the size of training set $\displaystyle S$. Occam's Razor's principle is considerable: "If you have two equally likely solutions to a problem, choose the simplest". Therefore we should choose $\displaystyle F$ which is the simplest one among the classes of function whose size of parameter is correspondant with the size of $\displaystyle S$, in order to prevent both \textit{underfitting} and \textit{overfitting} problem.


\section{Optimization}
Optimization is included in deep learning algorithms, from supervised to unsupervised learning. In the last section, we see that we can reduce the task of learning a model for a supervised problem to solving an optimization problem of the form \eqref{equation_loss_in_train_set_of_f_theta}, where $\displaystyle \theta$ is a parameter vector and $\displaystyle E_{(x,y) \in S}[L(f_{\theta}(x), y)$ is the average loss of all examples in training set $\displaystyle S$. 

We know that it is too difficult to solve this optimization task by direct method, which is solving the system of first order equations. Therefore an indirect method is popularly chosen. It is called gradient descent. This method consists of two steps alternatively repeated until finding out an adequately suitable parameter $\displaystyle \theta$: the first one is evaluating $\displaystyle f(\theta)$ on the training set $\displaystyle S$ and the second is updating $\displaystyle \theta$ by taking a small steps in the negative direction of the gradient $\displaystyle \nabla_{\theta}L$. These two steps can be represented by the pseudo code below:
\begin{algorithm}
    \caption{Gradient Descent}
    \begin{algorithmic}[1]
        \State $\theta_0 \gets \textit{Initialize}$
        \For{\texttt{iterations}}:
            \State $\theta_{t + 1} \gets \theta_{t} - \nabla F(\theta_{t}) . \epsilon$
            \State $t \gets t + 1$
        \EndFor
    \end{algorithmic}
\end{algorithm}

However, gradient descent method has some disavantages that make it becomes hard to be applied in reality. The first is that it highly depends on initialization and leads to a stuck in local optima. The second is that it needs a careful selection of learning rate parameter. Time to wait for one update is correlated with the number of examples in training set, which may be more than millions, so if the parameter $\displaystyle \theta$ is not updated in a right way, we need to restart the training process and waste a lot of useless time. 

In order to improve gradient descent, many gradient-based methods are invented, such as Stochastic Gradient Descent (SGD), Momentum, Adadelta, Adagrad, Adam, \dots. In SGD, we only take a small minibatch of training examples, which is suitable for our hardware resource, at a time and update parameter $\displaystyle \theta$ using these examples. The pseudo code of SGD is summaried in the following algorithm.
\begin{algorithm}
    \caption{Stochastic Gradient Descent}
    \begin{algorithmic}[1]
        \State $\theta_0 \gets \textit{Initialize}$
        \Repeat
            \State Sample a minibatch of $\displaystyle m$ of examples $\displaystyle \{(x_1, y_1), \dots, (x_m, y_m)\}$ of $\displaystyle S$
            \State Estimate the gradient $\displaystyle \nabla_{\theta}F(\theta) \approx \nabla_{\theta}[\frac{1}{m}\sum_{m}^{1}{L(F(x_i, \theta),y_i)}]$ with backpropagation
            \State Perform a parameter update: $\displaystyle \theta \gets \theta - \epsilon . \nabla_{\theta}F(\theta)$
        \Until{convergence}
    \end{algorithmic}
\end{algorithm}

SGD tends to 

Momentum
Adagrad
Adam

\section{Backpropagation}

\section{Neural Networks}
\subsection{Feedforward Neural Networks}
The Feedforward Neural Network (FNN) is the most basic type of neural networks. A FNN consists of a number of layers of artificial neurons which represent how input data are transformed to produce desired output. This type of network is called \textbf{Feedforward} because it is able to be considered as a flow from input $\displaystyle x$, through some intermediate computations, and finally to output $\displaystyle y$. The information is transfered in only one way, from layer $\displaystyle i$ to layer $\displaystyle i + 1$, without feedback from deeper layers to the previous layers. A layer is typically the composition of these operations: matrix multiplication and activation function. For example, a 2-layer FNN can be formulated as $\displaystyle y = W_2(\sigma(W_1(x)))$, $\displaystyle W_1, W_2$ are real-number matrices and $\displaystyle \sigma$ is a non-linear activation function, such as $\displaystyle tanh$, $\displaystyle sigmoid$, $\displaystyle relu$, \dots.

Formally, a Feedforward Neural Network with $\displaystyle l$ hidden layers (layers between input and output layer) can be parameterized by $\displaystyle l + 1$ weight matrices $\displaystyle (W_0, W_1, \dots, W_l)$, $\displaystyle l + 1$ scalar bias $\displaystyle (b_0, b_1, \dots, b_l)$ and $\displaystyle l$ activation functions $\displaystyle \sigma_1, \dots, \sigma_l$. Let $\displaystyle x$ is input, the feedforward neural networks is computed as follow:
\begin{algorithm}
    \caption{Feedforward Neural Network}
    \begin{algorithmic}[1]
        \State $z_0 \gets x$
        \For{i \textbf{from} $1$ \textbf{to} $l + 1$}
            \State $x_i \gets W_{i - 1}z_{i - 1} + b_i$
            \State $z_i \gets \sigma_i(x_i)$
        \EndFor
    \end{algorithmic}
\end{algorithm}

A typical example of FNN is the XOR problem. XOR function is the operation of two binary values, $\displaystyle x_1$ and $\displaystyle x_2$. The result of XOR function is $\displaystyle 1$ if $\displaystyle x_1$ and $\displaystyle x_2$ are equal, and is $\displaystyle 0$ otherwise. To solve this problem by neural networks, we first consider a single-layer perceptron ($\displaystyle y = \sigma(W_1x + b_1)$). After a time of trying to fit this network, we get a stuck that the this single-layer network can only approximate linearly separable functions. However, XOR function isn't linearly separable (see fig %add figure here
). Now let consider a 1-hidden-layer FNN with activation function relu ($\displaystyle relu(x) = max(0, x)$), we can now specify XOR function as:
\[ f(x; W_1, W_2, b_1, b_2) = W_2.max(0, W_1x + b_1) + b_2 \label{equation_xor} \]
We can now specify a solution to the XOR problem. Let
\[ W_1 = 
\begin{bmatrix}
    1 & 1 \\
    1 & 1
\end{bmatrix}, \]

\[ b_1 =
\begin{bmatrix}
    0 \\
    -1
\end{bmatrix}, \]

\[ W_2 =
\begin{bmatrix}
    1 & -2
\end{bmatrix}, \]

and $\displaystyle b_2 = 0$. We can check the result by computing the value of \eqref{equation_xor} with the given input $\displaystyle \tX$ below:
\[ \tX = 
\begin{bmatrix}
    0 & 0 & 1 & 1 \\
    0 & 1 & 0 & 1
\end{bmatrix}, \]
each column of $\displaystyle \tX$ is an input pair $\displaystyle (x_1, x_2)$,
\begin{align*}
    f(\tX; W_1, W_2, b_1, b_2) &= W_2.max(0, W_1x + b_1) + b_2 \\
    {} &= \begin{bmatrix}
            1 & -2
          \end{bmatrix}.max \left(0, 
            \begin{bmatrix}
                1 & 1 \\
                1 & 1
            \end{bmatrix} \begin{bmatrix}
                0 & 0 & 1 & 1 \\
                0 & 1 & 0 & 1
            \end{bmatrix} + \begin{bmatrix}
                0 \\
                -1
            \end{bmatrix}\right) + 0 \\
    {} &= \begin{bmatrix}
        1 & -2
      \end{bmatrix}.max \left(0, 
        \begin{bmatrix}
            0 & 1 & 1 & 2 \\
            0 & 1 & 1 & 2
        \end{bmatrix} + \begin{bmatrix}
            0 \\
            -1
        \end{bmatrix}\right) \\
    {} &= \begin{bmatrix}
        1 & -2
        \end{bmatrix}.max \left(0, 
        \begin{bmatrix}
            0 & 1 & 1 & 2 \\
            -1 & 0 & 0 & 1
        \end{bmatrix}\right) \\
    {} &= \begin{bmatrix}
        1 & -2
        \end{bmatrix}.\begin{bmatrix}
            0 & 1 & 1 & 2 \\
            0 & 0 & 0 & 1
        \end{bmatrix}\\
    {} &= \begin{bmatrix}
        0 & 1 & 1 & 0
        \end{bmatrix}
\end{align*}
We can see that the final result is exactly what we expected.


\subsection{Convolutional Neural Networks}
Convolutional Neural Networks (CNNs) are a specialized kind of feedforward neural networks for processing data that has a grid-like topology. This type of networks archives a considerable amount of success in many machine learning problem, from 1D data (time-series audio signal,\dots) to 2D data (computer vision,\dots). The most noticeable difference between CNNs and FNNs is that instead of using a matrix multiplication as FNN, CNNs use convolution operation. In the context of this thesis, we concentrate on the 2D convolution operation which is executed by two operators: a matrix which is the restricted portion of the receptive field and a matrix which is the set of learnable parameters (a.k.a. kernel). The kernel is spatially smaller than an image, but has the same number of channels. For example, if the image is composed of three (RGB) channels, the kernel height and width will be spatially small, but the depth extends up to all three channels. Usually, there is more than one kernel in the same convolutional layer, e.g. 16, 32, 64 kernels.

\textbf{Concrete Example.} Suppose that we have a input image $\displaystyle \tX$ with size \newline $\displaystyle 32 \text{ x } 32 \text{ x } 3$, a kernel $\displaystyle w$ with size $\displaystyle 5 \text{ x } 5 \text{ x } 3$, padding isn't used and stride is $1$. This kernel has $\displaystyle 5 * 5 * 3 = 75$ parameters to be learnt. We can convolve this filter by sliding it across all spatial positions of the input tensor and computing a dot product between a small chunk of $\displaystyle \tX$ and the filter $\displaystyle w$ at each position. The result will be an activation map, which in this case would have the dimensions $28 \text{ x } 28$ (28 is the number of unique positions that a filter of 5 elements can be placed over an input of size 32). In some cases, padding is used when users want to increase the spacial dimensions of the output (commonly the same as input tensor $\displaystyle \tX$). In addition, if there is no padding and stride 2 is used, output tensor has size of $\displaystyle 14 \text{ x } 14$. Finally, if kernel size is set to 64, input tensor $\displaystyle \tX$ executes convolution operation independently with these filters, and the final output has the shape of $\displaystyle 28 \text{ x } 28 \text{ x } 64$.

\textbf{General definition.} Formally, a convolutional layer for images (assuming input tensors with three spatial dimensions):
\begin{itemize}
    \item Has a $\displaystyle W_1 \text{ x } H_1 \text{ x } D_1$ tensor as input.
    \item Requires \textbf{4 hyperparameters}: The number of filter $\displaystyle K$, the spatial dimension $\displaystyle F$, the stride $\displaystyle S$, and the amount of padding on the borders of the input, $\displaystyle P$.
    \item The number of parameters of each filter is $\displaystyle F \text{ x } F \text{ x } D_1$ and one bias (if used bias), so with a set of $\displaystyle K$ filters, there is totally $\displaystyle K \text{ x } F \text{ x } F \text{ x } D_1$ weights and $\displaystyle K$ biases.
    \item The output is a tensor of size $\displaystyle W_2 \text{ x } H_2 \text{ x } K$, where $\displaystyle W_2 = (W_1 - F + 2P)/S + 1$, $\displaystyle H_2 = (H_1 - F + 2P)/S + 1$.
\end{itemize}

% Add figure of convolution operation in Deep Learning Book

\section{Recurrent Neural Networks}
\subsection{Vanilla Recurrent Neural Networks}


\subsection{Long-Short Term Memory Neural Networks}

\subsection{Convolution Long-Short Term Memory Neural Networks}


\section{Summary}
