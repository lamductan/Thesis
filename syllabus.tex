\chapter{Đề cương chi tiết}
\begin{longtable}{|l|c|}
\hline
\multicolumn{2}{|m{\linewidth}|}{\textbf{Tên đề tài}: Nhận dạng và truy vấn đối tượng ba chiều với Ring View và Neural Embedding
}\\
\hline
\multicolumn{2}{|m{\linewidth}|}{\textbf{Giảng viên hướng dẫn}: PGS.TS. Trần Minh Triết - ThS. Nguyễn Vinh Tiệp} \\
\hline
\multicolumn{2}{|m{\linewidth}|}{\textbf{Thời gian thực hiện}: 02/01/2018 - 30/06/2018}\\
\hline
\multicolumn{2}{|m{\linewidth}|}{\textbf{Sinh viên thực hiện}:Bùi Ngọc Minh (1412314) - Đỗ Trọng Lễ (1412673)}\\
\hline
\multicolumn{2}{|m{\linewidth}|}{\textbf{Loại đề tài}: nghiên cứu lý thuyết, đề ra mô hình cải tiến.}\\
\hline
\multicolumn{2}{|m{\linewidth}|}{\textbf{Nội dung đề tài} (mô tả chi tiết nội dung đề tài, yêu cầu, phương pháp thực hiện, kết quả đạt được, …):\par
Mục tiêu đề tài nhằm nghiên cứu các phương pháp đang được áp dụng trong bài toán nhận dạng và truy vấn đối tượng ba chiều, từ đó đưa ra một mô hình cải tiến.\par
Nội dung thực hiện chi tiết bao gồm:
\begin{itemize}
\item Tìm hiểu yêu cầu bài toán nhận dạng và truy vấn đối tượng ba chiều, các vấn đề trong lĩnh vực này.
\item Tìm hiểu các cách thức biểu diễn dữ liệu ba chiều trên máy tính. Tìm hiểu các tập dữ liệu thường được sử dụng trong lĩnh vực nhận dạng và truy vấn đối tượng ba chiều.
\item Tìm hiểu các phương pháp trong vấn đề nhận dạng và truy vấn đối tượng ba chiều. Tìm hiểu các hướng nghiên cứu chính trong lĩnh vực này.

\item Tìm hiểu kiến trúc của các mạng Deep Neural Networks, cách cài đặt các mạng Neural Networks trên các framework như Tensorflow và PyTorch.

\end{itemize}}\\
\hline
\multicolumn{2}{|m{\linewidth}|}{\begin{itemize}

\item Tìm hiểu các phương pháp Neural Embedding và cách áp dụng Neural Embedding vào các loại hình dữ liệu khác nhau và vận dụng vào bài toán truy vấn đội tượng ba chiều.

\item Đề xuất cải tiến sử dụng kiến trúc mạng Ring View Network dựa trên các phương pháp phân lớp dựa trên multi-view.

\item Thiết kế hệ thống camera ảo sao cho hợp lý và có thể thực hiện tốt trên cơ sở dữ liệu nhóm đã chọn.

\item Tiến hành thí nghiệm các mô hình đã đề xuất. Đánh giá kết quả của các mô hình. Từ đó chọn ra mô hình hiệu quả.
\end{itemize}}\\
\hline
\multicolumn{2}{|m{\linewidth}|}{
\textbf{Kế hoạch thực hiện}:
\begin{itemize}
\item 02/01/2018-15/01/2018: Tìm hiểu bài toán nhận dạng và truy vấn đối tượng ba chiều, các vấn đề liên quan.
\item 16/01/2018-01/02/2018: Tìm hiểu cách thức biểu diễn dữ liệu ba chiều trên máy tính. Tìm hiểu các tập dữ liệu được sử dụng. Các phương pháp chuyển đổi, chuẩn hóa dữ liệu.
\item 01/02/2018-15/02/2018: Nghiên cứu về các phương pháp phân lớp dựa trên multi-view.
\item 15/02/2018-28/02/2018: Nghiên cứu Neural Embedding và các kĩ thuật truy vấn thông tin.
\end{itemize}}\\
\hline
\multicolumn{2}{|m{\linewidth}|}{
\begin{itemize}
\item 01/03/2018-18/03/2018: Cài đặt các hệ thống camera ảo và chụp hình view ảo bằng OpenGL.Trích xuất đặc trưng của các view ảo bằng mạng học sâu ResNet50.
\item 19/03/2018-15/04/2018: Cài đặt RV-Network với các kiến trúc mạng khác nhau và cấu hình camera ảo khác nhau.
\item 16/04/2018-13/05/2018: Cài đặt Neural Embedding cho dữ liệu ba chiều
\item 14/05/2018-31/05/2018: Thử nghiệm các cài đặt RV-Network trên các tập dữ liệu.
\item 01/06/2018-15/06/2018: Thử nghiệm các cài đặt Neural Embedding trên các tập dữ liệu.
\item 16/06/2018-30/06/2018: Hoàn tất khóa luận.
\end{itemize}}\\
\hline
\makecell{\textbf{Xác nhận của GVHD} \vspace*{3cm}} & \makecell{\textbf{Ngày 26 tháng 12 năm 2017}\\ \textbf{Nhóm SV thực hiện} \vspace*{2cm} \\Bùi Ngọc Minh - Đỗ Trọng Lễ}\\
\hline
\end{longtable}