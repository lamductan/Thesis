\begin{EnAbstract}

Thông tin, dữ liệu dạng ba chiều ngày càng trở nên quan trọng trong những năm trở lại đây, với sự phát triển của các kĩ thuật cũng như nhu cầu thực tế, chẳng hạn sự xuất hiện của các camera ba chiều giá rẻ có thể tích hợp trong thiết bị di động như điện thoại hoặc các nhu cầu về xe tự lái, người máy tự hành cũng như chứng thực khuôn mặt có sử dụng thông tin độ sâu đẩy mạnh các nghiên cứu liên quan tới lĩnh vực ba chiều. Mở rộng từ lĩnh vực thị giác máy tính trong lĩnh vực ảnh hai chiều, lĩnh vực thị giác máy tính trên dữ liệu ba chiều cũng tập trung giải quyết các bài toán nền tảng như nhận dạng và truy vấn đối tượng ba chiều. Các phương pháp truyền thống có sẵn, được nâng cấp từ lĩnh vực ảnh hai chiều như đặc trưng SIFT3D cũng như các đặc trưng thiết kế riêng cho lĩnh vực ba chiều như RoPS, PFH. Các phương pháp truyền thống này có đặc điểm phải được lựa chọn phù hợp cho từng loại dữ liệu, nghĩa là không mang tính tổng quát hóa cho tất cả mọi loại dữ liệu. Hiện nay, xu hướng giải quyết vấn đề nhận dạng và truy vấn đối tượng ba chiều tập trung vào việc sử dụng các kiến trúc mạng neural network, cũng như tìm ra phương pháp để áp dụng deep neural networks trên dữ liệu ba chiều một cách hiệu quả như trên ảnh hai chiều.

Trong khóa luận, nhóm sinh viên tìm hiểu một cách chi tiết các cách thể hiện thông tin dữ liệu ba chiều trên máy tính, các tập dữ liệu liên quan tới chủ đề thị giác máy tính trên dữ liệu ba chiều, các phương pháp hiện tại để giải quyết vấn đề nhận dạng và truy vấn dữ liệu ba chiều. Nhóm sinh viên đề xuất giải pháp dựa trên nhóm các phương pháp về multi-view, với ý tưởng sử dụng nhiều ảnh hai chiều để mô tả một đối tượng ba chiều. Nhóm sinh viên đề xuất các cách thiết lập vị trí camera để render các view, các cách tổ hợp các view thành một ring và các cách tổ hợp các view, các ring. Nhóm xây dựng một mạng neural network dựa trên kĩ thuật transfer learning từ đặc trưng học sâu bởi các mạng Deep Neural Networks, dùng để phân lớp các Ring View gọi là RV-Net. Sự khác biệt trong so với các phương pháp hiện tại là hệ thống ring view có khả năng duy trì thứ tự giữa các view cục bộ trong một ring, tuy nhiên không hạn chế việc các ring với nhau cần có một thứ tự cố định. Điều này giúp cho mạng tăng khả năng bất biến với việc đối tượng có nhiều hướng. Ngoài ra, việc thiết kế các ring cũng có khả năng nhấn mạnh những phần cần được chú ý nhiều hơn trong một đối tượng không toàn vẹn. Với phương pháp RV-Net, nhóm sinh viên đã tiến hành thí nghiệm trên các tập dữ liệu từ cuộc thi RGB-D to CAD SHREC2017, SHREC2018 và có kết quả tốt nhất so với các nhóm nghiên cứu khác, đồng thời đạt độ chính xác 91.13\% trên tập dữ liệu ModelNet40.

Nhóm sinh viên cũng trình bày một giải pháp dựa trên Neural Embedding mà cụ thể là mô hình Paragraph Vector để xây dựng mô hình View Ring Vector biểu diễn đối tượng ba chiều dưới dạng vector. Từ biểu diễn vector này, nhóm sinh viên thực hiện truy vấn để tìm nhãn phân lớp của đối tượng. Sau tiến hành cài đặt và thí nghiệm, mô hình với cấu hình tốt nhất cho kết quả độ chính xác trên tập SHREC17 là 75.18\%. Tuy không tốt hơn phương pháp RV-Net được nhóm sinh viên đề xuất nhưng cũng cho thấy được tiềm năng của phương pháp này có thể cải tiến và đạt được kết quả tốt hơn. 
% Nhóm sinh viên cũng trình bày một giải pháp dựa trên Neural Embedding để áp dụng ý tưởng Neural Embedding, cụ thể là ..., trong việc phân lớp và truy vấn đối tượng ba chiều. Nhóm sinh viên tiến hành các thí nghiệm để cấu hình ...,...


\end{EnAbstract}